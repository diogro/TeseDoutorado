\pagestyle{empty}
\cleardoublepage
\pagestyle{fancy}

\selectlanguage{english}


\chapter{How does modularity in the genotype-phenotype map shape development and evolution?}\label{oqya}

Diogo Melo

\newpage

\vspace*{10pt}
% Abstract
\begin{center}
  \emph{\begin{large}Abstract\end{large}}
\vspace{2pt}
\end{center}

\noindent
Traits do not evolve independently, as genetic and
developmental associations affect the variation that is expressed in
populations and that is available for evolutionary change. In this
chapter, we explore the causes and consequences of structured variation,
introducing the concept of modularity, exploring some possible causes
for modular organization in different levels, and finally, discuss how
the introduction of new variation can evolve
\par
\vspace{1em}
\noindent\textbf{Keywords:} G-matrix; QTL mapping; genome prediction; genetic architecture
\newpage

\begin{refsection}

\subsection{Evolution and variation}\label{evolution-and-variation}

\begin{quote}
``Hence if man goes on selecting, and thus augmenting, any peculiarity,
he will almost certainly modify unintentionally other parts of the
structure, owing to the mysterious laws of correlation.'' --
\textbf{\textcite{Darwin1872}}
\end{quote}

Evolution proceeds by many different processes, all of which depend on
the variation present in natural populations. The probability of
fixation or loss of a neutral variant due to drift depends on its
frequency in a population. The increase or decrease via natural
selection of the frequency of an allele that has an effect on fitness
depends on the standing variation in that locus. Therefore, the fate of
a new variant depends on the population in which the new variant
appears, whether it is neutral or not. Advantageous variants that are
quite frequent may be lost in small populations, while even the smallest
advantage in fitness can guarantee that a rare variant will be fixed in
very large populations. In an analogous way, the change in the
distribution of a phenotype in a population depends on its standing
variation, and the details of this variation can profoundly alter the
evolutionary process. For example, consider a hypothetical selection
regime that operates as to increase the length of the left arm of the
individuals in a population. Individuals that have a long left arm leave
more offspring, and if the trait is heritable, the offspring of these
individuals have themselves longer left arms. So, the mean length of the
left arm increases due to natural selection, but the right arm is seldom
very different from the left arm, and so these long-left-armed
individuals also have long right arms, and the mean of the right arm
also changes between generations, due to a process we now recognize as
correlated response to selection \parencite{Lande1983-ez}. This indirect
correlated response happens because left and right arms are genetically
correlated, and when the genes for long-left-arm are passed on to the
next generation in large numbers, they are also genes for
long-right-arm, resulting in an evolutionary change in a structure that
was not under selection. This example illustrates that, while
individuals are composed of multiple parts, these parts are not
independent, and how they are related in a population can indeed alter
evolutionary outcomes. This is a problem familiar to all animal breeders
throughout history \parencite{Hazel1943-uq}, as attempts to selectively breed
individuals in order to improve some aspect of the populations
invariably leads to changes to other aspects. This problem suggests that
a more complete understanding of diversification and evolution must
somehow include a model for the evolution of the relations between
traits.

Phenotypes are always a complicated affair. Even the simplest of
biological structures is composed of several different parts and must be
created from scratch in each individual, requiring several thousand
proteins and molecules to interact in some way. All the different traits
in an individual must be formed through development, and the blueprints
for this process is encoded in the genome. So, if a phenotype is formed
by development, then it must follow that all the variation in these
phenotypes in a population must also be created, or at least mediated,
by development. There are several sources of variation in biological
populations. Some proportion of the variation in a population will be
due to differences in the genetic makeup of individuals, while some
other part will be due to variation inherent to the development process,
some part will be due to environmental differences between individuals,
and so on. Because all these sources of variation must be channeled
through development, the structure of development imposes constraints in
the variation that is ultimately expressed in a population. If the left
and right arm share, to some degree, some developmental pathways,
variation in these pathways will lead to correlated variation in both
arms. The path between genetic variation and phenotypic variation is
crucial and can be expressed in the genetic architecture.

\subsection{Genetic architecture and the genotype-phenotype
map}\label{genetic-architecture-and-the-genotype-phenotype-map}

Genetic architecture is
the structure of the relation between genotype and phenotype
\parencite{Hansen2006-ct}. Genetic variation acts through some sort of
developmental process to produce variation in the phenotype. But not all
genetic variation will affect the full phenotype of the individual. Some
parts of the genome only affect very specific phenotypes, while others
will have generalized effects in the whole organism. Genetic
architecture defines this relation and can have important consequences
to the evolutionary process, while at the same time being modified by
it.

One way of understanding genetic architecture is by constructing
genotype-phenotype maps (GP maps), which are mappings between genetic sequences and
phenotypes. Here, phenotypes can refer to a broad range of traits, from
exceedingly simple to the mind-bogglingly complex. Perhaps the simplest
possible example of a GP map that is still relevant in biology is the
secondary structure of an RNA molecule. For a given RNA molecule
sequence (genotype), the very simple development of folding the molecule
results in a particular shape (phenotype) \parencite{Ancel2000-vt}. This
shape is uniquely determined by the sequence, but different parts of the
sequence interact in complicated ways to generate the full final shape.
Because of the richness of the relation between sequence and shape, this
simple system has been extensively used as a model of evolution
\parencite{Stadler2006-og}, and indeed presents several properties that we
aim to understand in more complex phenotypes, such as modularity and
robustness. We can also think of the genetic architecture and GP map of
more complex traits, such as behavior, gene expression, skeletal
structures, growth, body shape, body composition and so on. Evolution of
these complex traits, composed of several interacting parts, is
profoundly influenced by genetic architecture.

We often rely on mathematical models to explore the consequences of
assumptions on the structure of the GP map. One of the earliest models
of complex phenotypic change, Fisher's geometric model
\parencite{Fisher1930-bp}, already presented formidable consequences to
evolution \parencite{Orr2000-gn}. In this model, Fisher assumed a completely
pleiotropic genetic architecture. Pleiotropy is the situation where one gene affects more than one
unrelated trait, and in Fisher's model, all genes affected all traits.
An individual is represented as a point in some high dimensional
continuous phenotype space, with each dimension representing some trait
in the individual, and fitness is given by a selective surface with a
single optimum. Mutation is represented by a shift in all traits, and
this can be interpreted as a vector sum between the initial position,
and a small random vector representing mutation. This mode of mutation
implies complete pleiotropy, as every mutation can potentially affect
every trait. Furthermore, this geometric interpretation of mutation
gives the model its name. If the individual is not at the phenotypic
optimum, a mutation might increase or decrease fitness. If the mutation
increases fitness, it can be fixed via natural selection and lead to
adaptation, with probability proportional do the increase in fitness.
Under this model, \textcite{Kimura1983-qs}, and later 
\textcite{Orr2000-gn}, showed that the rate of adaptation decreases with
the number of traits, an effect called ``cost of complexity'', as more complex organisms would be
slower to adapt. This cost appears because, under complete pleiotropy,
only a small proportion of random mutations would alter all the traits
of an organism in a beneficial way, and most mutations will move the
individual away from the optimum. This result was shown to be fairly
robust to some possible mitigating assumptions regarding the genetic
architecture \parencite{Welch2003-mh}, and so posed a difficult mismatch
between observation and theory, as complex organisms composed of many
traits exist and seem to have no problem adapting to many environments,
sometimes with remarkable speed \parencite{Kinnison2001-lu}.

This paradox only became tractable in light of explicit tests of the
assumptions of the geometric model, namely the pattern of pleiotropy in
the GP map and its consequences to mutation. In the last few decades, we
have begun to experimentally explore the genetic architecture of complex
traits using molecular mapping techniques, which allows us to relate
genetic variation to phenotypic variation \parencite{Mackay2001-mk}.
Quantitative trait loci (QTL) studies and genome wide association
studies (GWAS) have allowed us to investigate which variants are related
to disease, to improve our agricultural efficiency, to develop optimal
breeding strategies, and to further our understanding of the
evolutionary process by directly assessing genetic architecture. Using
QTL mapping, Wagner and collaborators \parencite*{Wagner2008-oz} investigated
the assumptions the geometric model made with regards to the GP map and
showed that the assumptions of the geometric model are not reasonable.
This incongruence between data and Fisher's early model of genetic
architecture can be summarized by two key points. First, pleiotropy is
not global, and the vast majority of loci affect only a small number of
traits. Second, the pleiotropic effect of loci onto a trait does not
decrease with the loci's level of pleiotropy. In other words, there is a
non-trivial scaling of genetic effects with the degree of pleiotropy. It
turns out these details matter a great deal here, and these properties
of the genetic architecture of complex traits we were only recently able
to quantify experimentally are, therefore, fundamental for the
evolvability of complex organisms \parencite{Wagner2011-kp}. We may ask how
these evolvable genetic architectures came to be.

\subsection{Traits and modules}\label{traits-and-modules}

\begin{quote}
``It {[}adaptation of traits{]} can only be workable if both the
selection between character states and reproductive fitness have two
characteristics: continuity and quasi-independence. Continuity means
that small changes in a characteristic must result in only small changes
in ecological relations; a very slight change in fin shape cannot cause
a dramatic change in sexual recognition or make the organism suddenly
attractive to new predators. Quasi-independence means that there is a
great variety of alternative paths by which a given characteristic may
change, so that some of them will allow selection to act on the
characteristic without altering other characteristics of the organism in
a countervailing fashion; pleiotropic and allometric relations must be
changeable. Continuity and quasi-independence are the most fundamental
characteristics of the evolutionary process. Without them organisms as
we know them could not exist because adaptive evolution would have been
impossible.'' --- \textcite{Lewontin1979-iw}
\end{quote}

\begin{quote}
``But what are the structural features that make stepwise improvement
possible? The key feature is that, on average, further improvements in
one part of the system must not compromise past achievements (\ldots{})''
--- \textcite{Wagner1996-ui}
\end{quote}

The very existence of (\emph{semi})individualized traits depends on one
ubiquitous aspect of evolved genetic architecture and genotype-phenotype
maps: modularity. Modularity can
be defined very generally as a property of a system whose parts are
assembled into groups which are tightly associated, while maintaining a
relative independence between groups. In the context of biological
organisms, modularity appears at several levels of organization. Traits
are, in a sense, modules, recognizable units with relative independence,
and indeed Richard Lewontin went so far as to postulate that this
subdivision is fundamental to adaptation \parencite{Lewontin1979-iw}. The
cost-of-complexity paradox illustrates this nicely: without a genetic
architecture that provides some level of independence between traits,
adaptation can not occur. It is not surprising then that these
identifiable units we recognize as traits are modified during evolution
without severely affecting the rest of the organism, and, accordingly,
their independence is reflected in their genetic architecture. 
\textcite{Hansen2003-fh} points out that a modular genetic architecture is
not the only way to achieve independent traits, but the observed modular
organization at several levels of organization, from gene expression to
morphology, suggests a deep underlying principle of organisms
\parencite{Wagner2007-cx}.

Modularity occurs at several levels, and traits are organized into
larger modules, which may, for example, perform a given function or form
a structure. Olson and Miller \parencite*{Olson1958-qk} founded a research
project based on the holistic investigation of organisms and their
organization into interconnected groups and in their interrelations form
the whole individual. Their seminal book championed the idea of
morphological integration, which captures the varying degree of
interdependence between traits must poses in order to come together into
functional units and that can then perform the functions that are
required of them. Olson and Miller pointed out that we can identify
these groups of traits by their correlations, as traits in a functional
module should covary together, as a consequence of their mutual
requirements for the performance of a function. The importance of these
modules to evolution was elegantly posited by Wagner and Altenberg
\parencite*{Wagner1996-ui}, who brought the idea of a modular architecture as
a central concept of biological organization. These modules of
correlated morphological traits could then change with relative
independence during evolution, while the genetic correlations within
modules facilitate a coordinated response to selection, maintaining
their function if one of the elements within a module were to be altered
\parencite{Cheverud1984-mi, Cheverud1982-op}.

The advent of QTL mapping also allowed us to investigate how the
relative independence between these sets of traits related to the
genetic architecture. Studies in several levels of biological
organization show that the genetic architecture underlying these modules
is also modular, as the pleiotropic effects of genes are more often
restricted to traits within these groups. This is true of gene
expression \parencite{Hartwell1999-as, Segal2003-oq} all the way up to
morphological traits \parencite{Mezey2000-rs}. Modularity is also expressed
in development, as the several processes involved in the formation of a
given trait will also be relatively separate from one another and
conceptually different. To see this, we may return to the example of the
left and right arm. Both develop separately, and so in some sense, each
is formed by a separate developmental module, but both share a great
deal of genetic information and tend to evolve, and so are the same
evolutionary module. However, these relations are not static, as
evolutionary and developmental modules can be created or collapsed
during evolution. One familiar example of this kind of modular
reorganization is the case of the association between upper and lower
limbs in humans, which became less associated as a result of changes in
our mode of locomotion \parencite{Young2010-rm}. Changes in the pattern of
correlations between phenotypic traits suggest the underlying modular GP
map responsible for the genetic associations we observe in populations
can be altered by evolution. This realization has important
consequences, as we establish a feedback between selection and
associations. We now turn to the first part of this feedback.

\subsection{Evolution of modular GP
maps}\label{evolution-of-modular-gp-maps}

Advances in QTL mapping have allowed us to probe the genetic
architecture and describe the mechanistic basis for the evolution of
genetic architecture, and the origin of the genetic variation that can
allow changes in the GP map. Cheverud and colleges have shown that gene
interactions are a major source of variation in pleiotropic patterns
\parencite{Wolf2005-nr, Pavlicev2008-jy}. Epistasis, gene effects that
depend on the interaction between different loci, greatly enhances the
variational possibilities in natural populations by changing which loci
affects which traits. More importantly for our discussion, variation in
pleiotropic relations provide the necessary ingredient for us to
understand the evolution of modularity \parencite{Wagner1996-ui}. The link
between function and modules of phenotypic traits suggest that modular
organization is an adaptation, and so we focus on selective explanations
for the modular organization of genetic architecture. On the other hand,
plausible neutral mechanisms for the emergence of modularity in some
organizational levels have been proposed in the literature
\parencite{Wagner2007-cx, Lynch2007-kz}. Furthermore, since modularity is so
general and occurs at different levels of biological organization, it is
hard to imagine this property evolved by a single mechanism at all
scales, and therefore each kind of association might require different
explanations. Perhaps the ubiquity of modularity reflect these different
roads that lead to it. Therefore, how modularity evolves remains an open
question, and because all of these modular architectures at different
levels are already established in nature, the work on the possible
causes for its evolution relies heavily in mathematical and
computational models.

One of the important problems I have emphasized that is solved by
modularity is the need for robustness, in the sense that changes in one
part of the organism does not interfere with the others. Ancel and
Fontana \parencite{Ancel2000-vt} used a model for the secondary structure of
RNA molecules to study the origins of modularity, defined as the
independence between different parts of the RNA molecule in the process
of melting under increasing temperature. In a modular molecule sequence,
secondary structure is lost in the different parts of the molecule
(modules) independently, while in a non-modular sequence, the whole
molecule continually changes its configuration during melting. In their
simulations, stabilizing selection was applied to a population of
evolving sequences, based on their secondary structures. This selection
for robustness had a number of consequences in the GP map of the
sequences. Selected sequences were more robust to mutations, showed less
phenotypic variation at intermediary temperatures, there was a
convergence of phenotypic and genetic variation, and the selected
sequences also became more modular (i. e., conformations in different
parts of the molecule become more independent). This suggests that
direct selection for robustness can lead to the evolution of modularity,
but the increase in evolvability due to modularity is not present in
this simplified system, so the analogy breaks down somewhat. In any
event, it is quite possible that selection for robustness is a driver of
the evolution of modularity.

Selection for evolvability has
also been proposed as a possible cause of modularity. In quantitative
genetics, evolvability is defined as the available variation for the
response to selection, and \textcite{Pavlicev2011-wz}
proposed a model based on the existence of genetic variation for the
association between two traits. This variation was expressed in the form
of an additive Mendelian polymorphism for the correlation between two
traits in a population. Homozygous individuals for one allele contribute
to high correlation between the traits, while homozygotes for the other
allele show no correlation, and the heterozygotes show intermediary
correlations. Selection was modeled deterministically using the response
to selection equation from quantitative genetics theory
\parencite{Lande1979-by}. Under this model, selection for coordinated
evolution of the two traits (simultaneous increase or decrease in the
value of the traits) leads to the fixation of the allele encoding high
correlation, and corridor selection, when one trait is held constant and
the other traits is selected for increase, leads to the fixation of the
allele encoding low correlation. In these two scenarios, the allele that
provides the highest amount of variation in the direction of selection
is fixed, and so selection increases evolvability by either integrating
or separating trait variation. We observed a similar effect of
directional selection in a fully mechanistic model, where pleiotropy and
gene effects were allowed to change via mutation in a large population
of simulated individuals. Using this model, we were able to show in
\textcite{Melo2015-bk} that stabilizing selection and drift are not viable
candidates for the emergence of modularity in complex phenotypes
composed of many traits. Stabilizing selection was theoretically a
possible driver of modularity \parencite{Lande1980-kn, Cheverud1984-mi}, and
has been shown to be effective in a small number of traits
\parencite{Jones2007-xe, Jones2014-wj}, but the structure of high
dimensional variation prevents stabilizing selection from being
efficient for multiple traits. This difficulty appears because
stabilizing selection is very efficient at increasing within-module
correlations, but not efficient at reducing between-module correlations,
so modules can't form. We looked at the effect of directional selection
in the covariance structure and the pattern of pleiotropic relations
\parencite{Melo2016-yw}, we see that directional selection is a powerful
driver of modularity, and traits that are selected in the same direction
in the simulations rapidly become more associated than traits that are
selected in different directions. Also, we show that corridor selection
can create complex patterns of correlations, as traits under directional
selection become more associated within themselves, while traits under
stabilizing selection maintain an intermediate level of correlation, and
the correlation between these two groups is reduced. In all simulation,
the changes in the correlation structure are due to selective changes in
the GP map, in which pleiotropic relations are altered by selection,
increasing evolvability.

Moving to some non-morphological traits, selection for more than one
function has also been shown to promote modularity in gene regulation
networks, while single objective networks were more integrated
\parencite{Espinosa-Soto2010-tw}. This is somewhat analogous to the
continuous traits case we discussed above, where different parts of the
system become adapted to one function. These modular regulation networks
are also more stable and robust. Interestingly, when working with neural
networks, selection for multiple objectives was not sufficient for
creating modules in work done by \textcite{Clune2012-ha}. In their simulations, in
addition to the selection for multiple outputs, the neural networks only
became modular with the addition of a cost for connections between nodes
of the neural network. While only suggestive, this provides a possible
explanation to why modularity and not other pleiotropic organization
that provide evolvability {[}see \textcite{Pavlicev2011-xm} for examples{]} are
more common in nature: there could be a cost to maintaining high levels
of pleiotropy, even if not in the form of low evolvability.

All the models we've seen so far treat development as a black box that
does not influence modularity, which is clearly a rather strong
simplification. In an attempt to include the complications of
development, \textcite{Watson2014-pi} use an ingenious strong selection weak
mutation model that allow them to include explicit developmental
interactions to the GP map. In this model, both the initial (embryonic)
traits and the interactions between these traits in all phases of
development are under genetic control. At each step of development new
interactions add complexity to the final adult phenotype, and this adult
phenotype is exposed to selection regimes that can change every few
thousands of generations. Traits in this model tend to become more
associated throughout development when they are selected in the same
direction in all selection regimes, and become independent when they are
selected in different directions. Also, selection for different
independent modules can lead to developmental interactions that allow
composition of these modules to form novel morphologies that were not
the initial selected states, an emergent form of complex organization.
We now turn to these emergent properties of modularity that can
profoundly facilitate adaptation.

\subsection{Modular variability}\label{modular-variability}

\begin{quote}
``Evolvability is the genome's ability to produce adaptive variants when
acted upon by the genetic system. This is not to say that the variants
need to be `directed' (Foster and Cairns 1992) for there to be
evolvability, but rather, that they cannot be entirely `misdirected,'
that there must be some small chance of a variant being adaptive. The
situation is analogous to obtaining a verse of Shakespeare from monkeys
banging away on typewriters. Typewriters make this far more likely than
if the monkeys had pencil and paper. The type-writers at least constrain
them to produce strings of letters. Similarly, the genotype-phenotype
map constrains the directions of phenotypic change resulting from
genetic variation.'' -- \textcite{Wagner1996-ui}
\end{quote}

Perhaps the most interesting consequence of the modular structure of the
GP map and development is the effect this organization has on
variability. Günter Wagner has
often drawn the distinction between variation and variability
\parencite{Wagner1996-ui}. For our purposes, \emph{variation} refers to the
expressed differences between individuals in a given population: how
different are they, or how differences between individuals are
correlated. Using variation we might predict how a population evolves
under drift of natural selection, or make inferences regarding
variational modules. \emph{Variability}, on the other hand, refers to
the ability of the population to generate variation. Wagner likens
variability of an organism to the solubility of a substance. Solubility
does not refer to the physical state of being in solution, but instead
to a property that a given substance has that defines how it behaves
when in solution. Likewise, a population of genetically identical
individuals has no genetic variation, but still has variability, defined
by its mutational properties. (For example, new mutations could have
correlated effects on many traits due to shared development and genetic
architecture.) Variation present in populations that is available for
selection must ultimately come from mutation. We are often told that
mutation is random, but this is a rather strong simplification. In what
sense are mutations random? Dan Graur {[}\textcite{Graur2015-th}, pp. 34{]}
points out that mutations are not random with respect to genome position
or mutation type, and that mutational effect on fitness are species
specific, gender specific, developmental stage specific, and several
other non-random conditions. The only way in which mutations are random
is in that the probability of a given mutation is the same regardless of
whether it is advantageous, neutral, or deleterious in the individual in
which it appears \parencite{Luria1943-se}. The key point is that new mutation
can be structured by variability, and so new variation can also be
structured. In quantitative traits, we can describe and quantify
variability by using the mutational matrix, the covariance matrix of
mutational effects. This can be done experimentally using mutation
accumulation lines, measuring the correlation between phenotypic changes
that appear in these lines due to mutation. We expect that, under some
general conditions and given enough time, the genetic variation in a
population come to mirror the mutational matrix
\parencite{Lande1980-kn, Cheverud1984-mi, Jones2007-xe}. The form of the
mutational matrix, and of variability in general, depends on the GP map
and on development, as traits that share pleiotropic genes or
developmental pathways will be jointly altered by mutations. So, all the
results we have seen on selection altering GP maps have consequences to
variability and to the introduction of new variation in natural
populations.

Models for the evolution of the mutational matrix in quantitative traits
reveal the possibility for interesting dynamics. \textcite{Jones2014-wj} used an
individual based model with epistatic interaction to study the evolution
of the mutational matrix. Epistasis is important because it opens the
door for complex interactions, and can lead to variation in mutational
correlations. Under their model, the mutational matrix of two
quantitative traits evolves to match the selection surface matrix, and
so new mutations are biased by past selection. Consequently, variation
that is introduced by mutation tends to respect the past selective
surface, and if this surface is stable, new mutations have a lower
probability of being deleterious. This kind of reorganization of
variability also appears under directional selection in the model from
\textcite{Pavlicev2011-wz} and in \textcite{Draghi2008-cv}, which uses a different scheme
for the evolution of pleiotropic relations and trait associations.

While these mathematical and computational results are remarkable, they
are difficult to explore experimentally.
Epistasis and allele interactions
have been show to contribute significantly to the phenotypic covariation
in complex traits \parencite{Cheverud2004-qr, Wolf2005-nr, Wolf2006-xt, 
Pavlicev2008-jy, Huang2012-si}, but we still lack a deep
understanding on how this variation is explored by natural selection and
evolution. However, recently studies in natural populations and
artificial selection have begun to uncover the effects of selection on
covariation. Working with morphological skull traits, \textcite{Assis2016-vz} (in
natural populations) and \textcite{Penna2017-if} (in artificial selection
experiments) have shown that that variation can indeed be reorganized in
the direction of selection, increasing potential future evolvability,
the same kind of effect observed in simulations in \textcite{Pavlicev2011-wz} and
\textcite{Melo2015-bk}. Conversely, several studies have documented the opposite
effect, in which directional selection acts in a more traditional manner
in multivariate traits, eroding the genetic variance in the direction of
selection \parencite{Walsh2009-cn}. \textcite{Careau2015-sy} carefully documented this
effect in behavioral traits in mice using a large selection experiment,
in which response to selection plateaued after several generations of
selection. These differences might be explained by differences in the
genetic architecture of these two different types of traits, but more
detailed studies are certainly needed.

\subsection{Phenotypic space and concluding
remarks}\label{phenotypic-space-and-concluding-remarks}

This remarkable feedback between selection, variation and variability
suggests a deeper consequence of the structure of the GP map and
phenotypes. Most of our understanding and descriptions of phenotypes
assume that the space in which phenotypes exist is continuous,
Euclidean, and that we can measure how close two phenotypes are using a
natural distance measure. In this framework, we rely on carefully chosen
adaptive landscapes to explain why some portion of the phenotypic space
are not explored, and to account for the emergence of modularity. If not
for selection, this framework implicitly places no limitations on the
possible phenotypes of organisms. \textcite{Stadler2001-vt} provide a different
perspective, wherein phenotypic space is such that simple Euclidean
distances do not make sense (like the surface of Earth at large scales),
and phenotypes are not restricted only by selection, but also by
development and genetic architecture. In this space, distances depend on
genetic proximity and the GP map, thus limiting the set of paths that
the mean phenotype of a population can take. In this view, modularity
and robustness and several other unexplained complexities in phenotypic
evolution are a reflection of the underlying metric imposed by the GP
map. A simpler and less encompassing version of this idea was already
present in the quantitative genetics literature. For example,
\textcite{Lande1979-by} explicitly stated that a population's distance to an
adaptive peaks should be measured in genetic variance distance, not
morphological distance, and see \textcite{Steppan2002-be} and \textcite{Melo2016-yw} for an
exploration of the macroevolutionary consequences of this fact.

\printbibliography
\end{refsection}