% Limpa cabeçalhos.
% (solução para lidar com a númeração das páginas pré-textuais).
\pagestyle{empty}

%% Capa

\begin{titlepage}

% Se quiser uma figura de fundo na capa ative o pacote wallpaper
% e descomente a linha abaixo.
\ThisCenterWallPaper{0.65}{capa.png}
\addtolength{\wpXoffset}{0.78cm}

\begin{center}
\hspace*{15pt} {\LARGE \nomedoaluno}
\par
\vspace{40pt}
\hspace{15pt}  {\huge \titulo}
\par
\vspace{450pt}
\hspace*{10pt} {\Large São Paulo}\\
\hspace*{10pt} {\Large \the\year}
\end{center}
\end{titlepage}

% Faz com que a página seguinte sempre seja ímpar (insere pg em branco)
\cleardoublepage

% Numeração em elementos pré-textuais é opcional (ativada por padrão).
% Para desativá-la comente a linha abaixo.
\pagestyle{fancy}
\fancyhead{} % clear all fields
\fancyhead[LE,RO]{\thepage}

% Números das páginas em algarismos romanos
\pagenumbering{roman}

%% Página de Rosto

% Numeração não deve aparecer na página de rosto.
\thispagestyle{empty}

\begin{center}
{\LARGE \nomedoaluno}
\par
\vspace{120pt}
{\Huge \titulo}
\par
\vspace{80pt}
{\huge \tituloEN}
\end{center}
\par
\vspace{70pt}
\hspace*{175pt}\parbox{7.6cm}{{\large Tese apresentada ao Instituto de Biociências da Universidade de São Paulo, para a obtenção de Título de Doutor em Ciências, na Área de Biologia (Genética).}}

\par
\vspace{1em}
\hspace*{175pt}\parbox{7.6cm}{{\large Orientador: Gabriel Marroig}}

\par
\vfill
\vspace{10pt}
\begin{center}
\textbf{{\large São Paulo}\\
{\large \the\year}}
\end{center}

\cleardoublepage

% Ficha Catalográfica
\hspace{8em}\fbox{\begin{minipage}{10cm}
Melo, Diogo Amaral R.

\hspace{2em}\titulo

\hspace{2em}235 páginas

\hspace{2em}Tese (Doutorado) - Instituto de Biociências da Universidade de São Paulo. Departamento de Genética e Biologia Evolutiva.

\begin{enumerate}
\item arquitetura genética
\item covariação genética
\item evolução multivariada
\end{enumerate}
I. Universidade de São Paulo. Instituto de Biociências. Departamento de Genética e Biologia Evolutiva.

\end{minipage}}
\par
\vspace{2em}
\begin{center}
{\LARGE\textbf{Comissão Julgadora:}}

\par
\vspace{8em}
\begin{tabular*}{\textwidth}{@{\extracolsep{\fill}}l l}
\rule{16em}{1pt} 	& \rule{16em}{1pt} \\
Prof. Dr. 		& Prof. Dr. \\
Nome			& Nome
\end{tabular*}

\vspace{6em}

\begin{tabular*}{\textwidth}{@{\extracolsep{\fill}}l l}
\rule{16em}{1pt} 	& \rule{16em}{1pt} \\
Prof. Dr. 		& Prof. Dr. \\
Nome			& Nome
\end{tabular*}

\par
\vspace{6em}

\parbox{16em}{\rule{14em}{1pt} \\
Prof. Dr. \\
Gabriel Marroig}
\end{center}

\newpage

% Dedicatória
% Posição do texto na página
\vspace*{0.75\textheight}
\begin{flushright}
  \emph{Para Tomás e Rita.}
\end{flushright}

\newpage

% Epígrafe
\vspace*{0.4\textheight}
\noindent{\LARGE\textbf{}}
% Tudo que você escreve no verbatim é renderizado literalmente (comandos não são interpretados e os espaços são respeitados)
\begin{verbatim}
A human being should be able to change a diaper, 
plan an invasion, butcher a hog, conn a ship,
design a building, write a sonnet, balance accounts, 
build a wall, set a bone, comfort the dying, 
take orders, give orders, cooperate, act alone, 
solve equations, analyze a new problem, pitch manure, 
program a computer, cook a tasty meal, 
fight efficiently, die gallantly. 
Specialization is for insects.
\end{verbatim}
\begin{flushright}
Robert A. Heinlein
\end{flushright}

\newpage

% Agradecimentos

% Espaçamento duplo
\doublespacing

\noindent{\LARGE\textbf{Agradecimentos}}

\onehalfspacing
\noindent
Esse trabalho só foi possível com o apoio financeiro da FAPESP, que pagou minha bolsa e financiou o projeto associado, e da BBSRC-UK, que financiou o projeto na Inglaterra. Também foram essenciais o espaço e o apoio institucional da Universidades de São Paulo e da Universidade de Bath. Especialmente toda a equipe do biotério de Bath, e toda a equipe administrativa e de zeladoria das duas universidades. 

Durante este doutorado eu contei com dois orientadores e não poderia pedir uma dupla melhor. Primeiro o oficial, Gabriel, que me acompanha desde o começo da minha jornada científica, que me empurra pra ser melhor, e que durante os mais de 10 anos em que trabalhamos juntos deixou de ser de só um orientador para ser um grande amigo e colega. Foi um prazer fazer parte desse laboratório por tanto tempo, e acho que ainda temos muita estrada juntos. Segundo, o extra-oficial, Jason, que me recebeu na sua casa sem nem me conhecer, ocasião em que descobrimos rapidamente que somos o mesmo tipo de maluco. Meu tempo no seu laboratório foi um aprendizado em alta velocidade, acho que eu nunca aprendi tanto em tão pouco tempo. Foi extremamente divertido fazer esse trabalho com os dois, e o que está escrito nesta tese é só uns 20\% do que a gente fez! 

Ciência é um trabalho profundamente colaborativo, e aqui isso é muito claro. Todos os capítulos têm co-autores que provavelmente trabalharam mais do que eu. Anna, Eleanor e Arthur, valeu pela força e pela companhia, espero que sejam os primeiros de muitos trabalhos juntos. Também tenho que agradecer a Valentina, pelo trampo hercúleo com os camundongos em Bath. Astral e Amanda, que continuaram esse trampo aqui, sem vocês não tinha a menor chance do projeto sair do lugar. Colegas do LEM e dos laboratórios do porão, valeu pela ajuda, pela companhia e por deixar esse trabalho muito melhor! Harley, Paulinha, Aninha, Dani, Tafinha, Alex, Lugar, Roberta, Fino,
Arthur, Leila, Monique, Wally, Pato, Edgar, Anna, Astral, Amanda, Jonatas, Vitor, Bárbara, Mica, Limão e todos os habitantes do porão, foi um prazer e um privilégio compartilhar esses anos com vocês. Agradecimento especial a Dani e Tafinha que me ajudaram a escolher o título da tese.

Eu passei boa parte do doutorado dedicado à ensinar e ajudar outros alunos como pudesse, e vários professores me fizeram a gentileza de aceitar minha ajuda com suas disciplinas. (Acho que por isso eu passei mais tempo nos corredores da ecologia do que do meu departamento…). Obrigado Alê e Paulo Inácio por me convidarem tantas vezes e por me ajudarem quando eu precisei; Glauco por me chamar para o curso de campo, que atingiu o seu singelo objetivo de mudar minha vida; Diogo e Gabriel pelos vários anos de processos; Roberto e Paulo Inácio pelas divertidíssimas escolas de verão. E todos meus eternos colegas nessas empreitadas: Ari, Sara, Ayana, Flávia, Paulinha, Pizza, Renatão, Chalom, Mali, Melina, Marília, Musgo, e mais um monte de gente entre o curso de R, modelos lineares, processos, SSSMB, curso de campo, bio evolutiva, curso de MCMC… enfim, muito curso! Agradecimento especial ao Galo, que segurou a monitoria do curso de campo comigo. Acho que um teria morrido sem o outro. Valeu também Diana, Luísa, Paula, Pinguim, Adrian, Irina, Billy, Edu, Mathias, Soli, Aspira, Rena, Rodrigo, e todo os outros alunos e professores do curso de campo.

A minha cientista social favorita me falou que se a pessoa é importante mesmo tem um parágrafo só pra ela nos agradecimentos. Não sei se eu concordo, mas fica registrado aqui um parágrafo só pra Mariana. Valeu querida, foi muito bom fazer um doutorado junto com você, e tenho certeza que só publiquei aquele artigo porque ir com você no bandeijão dá bastante sorte!

Obrigado também aos meus queridos amigos, que deixam a jornada mais leve. Agradecimento especial à Julia, que me abrigou e me fez companhia quando eu fui sozinho para a Inglaterra, PH, que me acompanhou desde a quinta série e agora estamos aqui publicando juntos (!), Thais, que estava sempre ai quando eu precisava discutir política com alguém, Anna, de novo, que mesmo longe continua sendo minha pessoa favorita. Um obrigado também pros compadres Janaína e Arthur, que além de tudo colocaram a Rosa na nossa vida. Tem gente demais nessa categoria, então sinta-se devidamente agradecido, mesmo se seu nome não estiver aqui. 

Por último, minha família, meus pais, irmãos, avós, primos, primas, sobrinhos e sobrinhas. Eu tenho a sorte de ter um monte de parentes adotivos fantásticos, então é bem mais gente do que parece (só de pais são 4, primos uns 50, sobrinho cada dia tem mais). E finalmente a Rita, meu grande amor, que me acompanhou o caminho todo, e com quem eu tive meu primeiro filho faltando 2 meses para entregar esta tese! Bem vindo, Tomás! Seus primeiros dois meses alegraram muito esse final de tese! 

Essa tese é pra vocês, espero que gostem.

\newpage

\doublespacing

\vspace*{10pt}
% Abstract
\begin{center}
  \emph{\begin{large}Resumo\end{large}}\label{resumo}
\vspace{2pt}
\end{center}
\noindent
Caracteres complexos são aqueles determinados por muitos genes e que apresentam variação contínua. Em uma população, a variação herdável dos caracteres complexos não é independente, e pares de caracteres podem ser mais ou menos correlacionados entre si. O nível e o padrão da associação entre caracteres determina como o fenótipo da população se comporta perante os processos evolutivos. A associação entre caracteres pode tanto facilitar a evolução em algumas direções do espaço fenotípico quanto restringir a evolução em outras, pois caracteres mais associados entre si tendem a evoluir de forma conjunta. O padrão de associação entre caracteres pode ser representado pela matriz de covariância genética aditiva, que descreve o padrão variacional resultante da interação do mapa genótipo-fenótipo e de todos os processos de desenvolvimento que levam desde a informação contida no material genético até o indivíduo. Tanto o mapa genótipo-fenótipo quanto o padrão de covariação genético também apresentam variação herdável, e portanto podem ser alterados pelos processos evolutivos e mudar entre gerações. Esse processo estabelece uma interação de mão dupla entre evolução e covariação, na qual a covariação afeta o resultado dos processos evolutivos e os processos evolutivos afetam a covariação. Nesta tese, nós exploramos como os efeitos genéticos interagem para formar o padrão de covariação, e como esses efeitos e covariação evoluem sob seleção natural. Para isso, nós trabalhamos com três populações experimentais de camundongos que foram sujeitas a regimes de seleção artificial e, utilizando diferentes tipos de caracteres, procuramos entender como a covariação se estabelece e como ela é afetada pela seleção. No primeiro experimento, estudamos o padrão de covariação de caracteres cranianos em linhagens selecionadas para aumento e diminuição do tamanho corporal, e observamos que a seleção para tamanho altera os caracteres do crânio e a covariação entre eles. A seleção direcional diminui a variação total do crânio, mas também aumenta a proporção de variação na direção de seleção, potencialmente facilitando uma nova resposta seletiva na mesma direção. Esse resultado implica que a variação presente em uma população pode ser moldada pela sua história evolutiva de forma adaptativa. No segundo experimento utilizamos uma população intercruzada, criada a partir linhagens selecionadas para aumento e diminuição do tamanho corporal, para identificar regiões genômicas envolvidas na determinação da curva de crescimento. Utilizando estimativas dos efeitos genotípicos nos fenótipos de crescimento, nós pudemos prever os fenótipos das linhagens ancestrais utilizando apenas informação da população intercruzada, e também construir estimativas de qual seria a covariação entre os caracteres de crescimento para cada tipo de efeito genético. Além disso, relacionamos a distribuição dos efeitos genéticos com a história evolutiva da população, mostrando que tanto a seleção quanto restrições internas do desenvolvimento interagem para determinar a distribuição de efeitos genéticos e, portanto, a covariação. No terceiro experimento, utilizamos seis linhagens de camundongos, que haviam sido selecionadas para alterações na curva de crescimento, para formar uma população intercruzada. Essa população apresentava uma enorme variação na sua curva de crescimento, e, utilizando técnicas de mapeamento genético, nós identificamos regiões genômicas envolvidas na determinação dessa variação fenotípica. Também desenvolvemos, para criar uma expectativa para a distribuição de efeitos genéticos nessa população, um modelo de simulação computacional da evolução dos efeitos genotípicos sob seleção. Os efeitos genéticos na população intercruzada apresentam um padrão mais complexo que o das simulações, e encontramos uma combinação de efeitos genéticos com padrões diferentes que interagem para gerar a covariação genética presente na população. Por fim, apresentamos uma revisão sobre a evolução da covariação genética e discutimos as consequências macroevolutivas das questões abordadas nos outros capítulos.
\par
\vspace{1em}
\noindent\textbf{Palavras-chave:} genética quantitativa, mapeamento de QTL, seleção direcional
\newpage

% Criei a página do abstract na mão, por isso tem bem mais comandos do que o resumo acima, apesar de serem idênticas.
\vspace*{10pt}
% Abstract
\begin{center}
  \emph{\begin{large}Abstract\end{large}}\label{abstract}
\vspace{2pt}
\end{center}

% Selecionar a linguagem acerta os padrões de hifenação diferentes entre inglês e português.
\selectlanguage{english}
\noindent
Complex traits are defined as traits that are determined by many genes and that show continuous variation. In a population, the heritable variation of complex traits is not independent, and pairs of traits might be more or less correlated. The level and pattern of the association between traits determine how the phenotype of the population behaves when faced with evolutionary forces, like natural selection and genetic drift. The association between traits can both facilitate evolutionary change in some directions of the phenotype space and hinder change in other directions because tightly associated traits tend to evolve together. The pattern of association among traits can be represented by the additive genetic covariance matrix. This matrix describes the variational pattern that is the result of the interplay between the genotype-phenotype map and development, which together lead from the genetic information to the formation of the individual. Both the genotype-phenotype map and the genetic covariation also show heritable variation, and so are able to evolve and change between generations. This process establishes a feedback between evolution and covariation, in which covariation affects the outcome of the evolutionary process and is also shaped by evolution. In this thesis, we explore how genetic effects interact to create patterns of covariation, and how these effects and covariation change under natural selection. In order to do this, we use three experimental mice populations that were subjected to artificial selection regimes, and, using several types of complex traits, we study how covariation is established and how it evolves. In the first experiment, we use the covariation pattern of cranial traits measured in mice strains selected for the increase and decrease of body size. In these strains, we see that size selection altered the means of the cranial traits and the covariation between them. Directional selection reduces the total amount of genetic information, but in a non-uniform way. Some directions in phenotype space lose more variation than others, and, counter-intuitively, the direction of selection loses less variation. This leads to an increase in the proportion of variation that is in the direction of selection, potentially facilitating future evolutionary change in the same direction. This result shows that the covariation pattern in a population is shaped by its evolutionary history and can be adaptive. In the second experiment, we use an intercross population, created with two inbred mouse strains that were selected for increase and decrease in weight, to identify genomic regions involved in determining the growth curve of the individuals. Using estimates of the genetic effects on the growth traits, we were able to predict the phenotypes of the ancestral strains using only information from the intercross. We were also able to partition the genetic covariation into the contributions due to different types of genetic effects. We interpret the distribution of genetic effects in light of the evolutionary history of the population and show that the distribution of genetic effects, and of genetic covariation, is a consequence of the interaction between selection and development. In the third experiment, we create an intercross using six inbred mice strains that had been selected for different changes in their growth curve. This intercross shows large variation in growth curves, and, using genetic mapping techniques, we identify genomic regions involved in producing this phenotypic variation. To create an expectation for the distribution of genetic effects in this population, we develop a computer simulation model for the evolution of genetic effects under directional selection. The genetic effects in the population are more complex than in the simulation model, and we find that the genetic covariation between growth traits is created by the interaction among several different kinds of genetic effects. Finally, we present a review on the evolution of genetic covariation and discuss the macroevolutionary consequences of the themes we explore in the other chapters.
\par
\vspace{1em}
\noindent\textbf{Keywords:} quantitative genetics, QTL mapping, directional selection

% Voltando ao português...
\selectlanguage{brazilian}

\newpage

% Desabilitar protrusão para listas e índice
\microtypesetup{protrusion=false}

% Lista de figuras
\listoffigures

% Lista de tabelas
\listoftables

% Abreviações
% Para imprimir as abreviações siga as instruções em 
% http://code.google.com/p/mestre-em-latex/wiki/ListaDeAbreviaturas
% \printnomenclature

% Índice
\tableofcontents

% Re-habilita protrusão novamente
\microtypesetup{protrusion=true}
