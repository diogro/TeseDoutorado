% Limpa cabeçalhos.
% (solução para lidar com a númeração das páginas pré-textuais).
\pagestyle{empty}

%% Capa
\begin{titlepage}

% Se quiser uma figura de fundo na capa ative o pacote wallpaper
% e descomente a linha abaixo.
\ThisCenterWallPaper{0.65}{capa.png}

\begin{center}
{\LARGE \nomedoaluno}
\par
\vspace{70pt}
{\Huge \titulo}
\par
\vspace{420pt}
\textbf{{\large São Paulo}\\
{\large \the\year}}
\end{center}
\end{titlepage}

% Faz com que a página seguinte sempre seja ímpar (insere pg em branco)
\cleardoublepage

% Numeração em elementos pré-textuais é opcional (ativada por padrão).
% Para desativá-la comente a linha abaixo.
\pagestyle{fancy}

% Números das páginas em algarismos romanos
\pagenumbering{roman}

%% Página de Rosto

% Numeração não deve aparecer na página de rosto.
\thispagestyle{empty}

\begin{center}
{\LARGE \nomedoaluno}
\par
\vspace{200pt}
{\Huge \titulo}
\end{center}
\par
\vspace{90pt}
\hspace*{175pt}\parbox{7.6cm}{{\large Tese apresentada ao Instituto de Biociências da Universidade de São Paulo, para a obtenção de Título de Doutor em Ciências, na Área de Biologia (Genética).}}

\par
\vspace{1em}
\hspace*{175pt}\parbox{7.6cm}{{\large Orientador: Gabriel Marroig}}

\par
\vfill
\begin{center}
\textbf{{\large São Paulo}\\
{\large \the\year}}
\end{center}

\newpage

% Ficha Catalográfica
\hspace{8em}\fbox{\begin{minipage}{10cm}
Melo, Diogo Amaral R.

\hspace{2em}\titulo

\hspace{2em}\pageref{LastPage} páginas

\hspace{2em}Tese (Doutorado) - Instituto de Biociências da Universidade de São Paulo. Departamento de Genética e Biologia Evolutiva.

\begin{enumerate}
\item arquitetura genética
\item covariação genética
\item evolução multivariada
\end{enumerate}
I. Universidade de São Paulo. Instituto de Biociências. Departamento de Genética e Biologia Evolutiva.

\end{minipage}}
\par
\vspace{2em}
\begin{center}
{\LARGE\textbf{Comissão Julgadora:}}

\par
\vspace{8em}
\begin{tabular*}{\textwidth}{@{\extracolsep{\fill}}l l}
\rule{16em}{1pt} 	& \rule{16em}{1pt} \\
Prof. Dr. 		& Prof. Dr. \\
Nome			& Nome
\end{tabular*}

\vspace{6em}

\begin{tabular*}{\textwidth}{@{\extracolsep{\fill}}l l}
\rule{16em}{1pt} 	& \rule{16em}{1pt} \\
Prof. Dr. 		& Prof. Dr. \\
Nome			& Nome
\end{tabular*}

\par
\vspace{6em}

\parbox{16em}{\rule{14em}{1pt} \\
Prof. Dr. \\
Gabriel Marroig}
\end{center}

\newpage

% Dedicatória
% Posição do texto na página
\vspace*{0.75\textheight}
\begin{flushright}
  \emph{Para Tomás e Rita.}
\end{flushright}

\newpage

% Epígrafe
\vspace*{0.4\textheight}
\noindent{\LARGE\textbf{}}
% Tudo que você escreve no verbatim é renderizado literalmente (comandos não são interpretados e os espaços são respeitados)
\begin{verbatim}
A human being should be able to change a diaper, 
plan an invasion, butcher a hog, conn a ship,
design a building, write a sonnet, balance accounts, 
build a wall, set a bone, comfort the dying, 
take orders, give orders, cooperate, act alone, 
solve equations, analyze a new problem, pitch manure, 
program a computer, cook a tasty meal, 
fight efficiently, die gallantly. 
Specialization is for insects.
\end{verbatim}
\begin{flushright}
Robert A. Heinlein
\end{flushright}

\newpage

% Agradecimentos

% Espaçamento duplo
\doublespacing

\noindent{\LARGE\textbf{Agradecimentos}}

Agradeço ao meu orientador, ao meu co-orientador, aos meus colaboradores, aos técnicos, à seção administrativa, à fundação que liberou verba para minhas pesquisas, aos meus amigos, à minha família e ao meu grande amor.

\newpage

\vspace*{10pt}
% Abstract
\begin{center}
  \emph{\begin{large}Resumo\end{large}}\label{resumo}
\vspace{2pt}
\end{center}
% Pode parecer estranho, mas colocar uma frase por linha ajuda a organizar e reescrever o texto quando necessário.
% Além disso, ajuda se você estiver comparando versões diferentes do mesmo texto.
% Para separar parágrafos utilize uma linha em branco.
\noindent
Esta, quem sabe, é a parte mais importante do seu trabalho.
É o que a maioria das pessoas vai ler (além do título).
Seja objetivo sem perder conteúdo.
Um bom resumo explica porquê este trabalho é interessante, relata como foi feito, o que foi encontrado, contextualiza os resultados e delineia conclusões.
\par
\vspace{1em}
\noindent\textbf{Palavras-chave:} palavra1, palavra2, palavra3
\newpage

% Criei a página do abstract na mão, por isso tem bem mais comandos do que o resumo acima, apesar de serem idênticas.
\vspace*{10pt}
% Abstract
\begin{center}
  \emph{\begin{large}Abstract\end{large}}\label{abstract}
\vspace{2pt}
\end{center}

% Selecionar a linguagem acerta os padrões de hifenação diferentes entre inglês e português.
\selectlanguage{english}
\noindent
This is the most important part of your work.
This is what most people will read.
Be concise without omitting content.
A good abstract explains why this is an interesting study, tells how it was done, what was found, contextualizes the results and set conclusions.
\par
\vspace{1em}
\noindent\textbf{Keywords:} word1, word2, word3

% Voltando ao português...
\selectlanguage{brazilian}

\newpage

% Desabilitar protrusão para listas e índice
\microtypesetup{protrusion=false}

% Lista de figuras
\listoffigures

% Lista de tabelas
\listoftables

% Abreviações
% Para imprimir as abreviações siga as instruções em 
% http://code.google.com/p/mestre-em-latex/wiki/ListaDeAbreviaturas
% \printnomenclature

% Índice
\tableofcontents

% Re-habilita protrusão novamente
\microtypesetup{protrusion=true}
