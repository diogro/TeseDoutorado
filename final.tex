\pagestyle{empty}
\cleardoublepage
\pagestyle{fancy}

\selectlanguage{brazilian}

%\pagenumbering{arabic}

\chapter{Considerações Finais}

\begin{refsection}

Nesta tese abordamos o problema da evolução e manutenção dos padrões de
covariância genética aditiva em caracteres complexos. Para isso, utilizamos
seleção artificial em diversos desenhos experimentais. Todos os regimes de
seleção artificial envolveram crescimento ou tamanho corporal. A complexidade
da definição do tamanho de um organismo garante que seleção neste caráter irá
ter influências indiretas em praticamente todos os caracteres morfológicos, e
portanto as consequências da seleção dependem criticamente da arquitetura
genética e do padrão de covariação. Neste capítulo final, vamos discutir
brevemente os resultados de cada um dos três capítulos experimentais e tecer
alguns comentários sobre as limitações e possíveis futuros desdobramentos do
nosso trabalho.

A influência generalizada no organismo da seleção para alteração do tamanho
corporal é bastante clara no capítulo 3. Neste capítulo, mostramos como
seleção para alterações no peso, tanto para aumento quanto diminuição, causam
respostas correlacionadas nos caracteres cranianos. A alteração coordenada dos
caracteres do crânio leva a uma mudança na sua estrutura da covariação
genética e fenotípica. A mudança na covariação genética tem dois aspectos
importantes. Primeiro, a quantidade de variação genética total diminui. Esse é
o tipo de mudança esperado sob seleção de curto prazo em caracteres de base
genética aditiva. A seleção provoca mudanças nas frequências alélicas dos loci
envolvidos na variação dos caracteres sob seleção. A medida que os alelos que
conferem maior aptidão aumentam de frequência e eventualmente se fixam, a
variância genética dos caracteres sob seleção diminui. Na seleção de
caracteres multivariados, a predição clássica para a mudança da covariância
sob seleção direcional é que o efeito seja o mesmo, com a redução da variação
longo da direção sob seleção. O livro texto clássico
\textcite{Falconer1996-ot} discute a mudança esperada na correlação entre dois
caracteres selecionados na mesma direção. A expectativa nesse caso é que, ao
final do processo seletivo, os alelos que influenciam os dois caracteres na
mesma direção estejam fixados, e que portanto apenas alelos com efeitos na
direção ortogonal à seleção estariam segregando em frequências intermediárias.
Como só haveria variação na direção ortogonal aos dois caracteres mudando
juntos, a correlação entre eles ao final do processo deveria ser negativa.
Isso é o exato oposto do que nós observamos no crânio dos camundongos
selecionados. A perda de variação total que nós observamos é completamente
anisotrópica, e algumas direções perdem mais variação do que outras. Mas, ao
contrario do esperado pelo modelo aditivo, a direção que perde menos variação
é justamente a direção de seleção. Proporcionalmente à variação total, a
seleção direcional aumenta a proporção de variação na direção de seleção. Nós
interpretamos esse resultado à luz da teoria recente de genética quantitativa
que inclui efeitos não aditivos na determinação do padrão de
covariação~\parencite{Pavlicev2011-wz, Cheverud2001-ho}. A inclusão de efeitos
epistáticos no nosso modelo da evolução da covariância genética da conta das
mudanças na covariância observadas nas linhagens selecionadas. Apesar disso,
nossa estimativa da matriz de covariância genética é bastante deficiente
devido à estrutura familiar das linhagens sob seleção, uma consequência do
desenho experimental que não foi feito com esse tipo de medida em
mente\footnote{O experimento de seleção artificial foi originalmente concebido
em um departamento de ciências veterinárias, e o objetivo do estudo era
avaliar a viabilidade das linhagens após seleção de longo prazo, com
aplicações para melhoramento genético bovino. Mérito do Gabriel em perceber a
possibilidade de aproveitar a colonia para o estudo da evolução da
covariação.}, e portanto boa parte dos resultados depende de algum nível de
similaridade entre as matrizes genéticas e fenotípicas, que parece razoável
dada nossa experiência com esses caracteres do
crânio~\parencite{Garcia2014-oj, Cheverud1988-he}. Além disso, não temos
nenhum dado de marcadores genéticos para esses animais, o que limita nossa
habilidade de confirmar a origem genética das mudanças na covariação. Mesmo
com essas limitações, esses resultados ilustram bem o tipo de mudança bastante
não intuitiva que seleção direcional pode ter na covariação genética e
fenotípica num intervalo de tempo relativamente curto.

O capítulo 4 utiliza um cruzamento clássico~\parencite[descrito
em][]{Cheverud1996-fm} de camundongos, selecionados para alterações no tamanho
corporal, para investigar o padrão pleiotrópico e a covariação genética entre
fases do crescimento. Este capítulo tem dois temas principais: (1) a
caracterização dos efeitos pleiotrópicos em relação à seleção e a divergência
fenotípica entre os fundadores; (2) a inferência dos padrões de covariação a
partir dos efeitos aditivos e de dominância. Cabe aqui um comentário sobre os
dados utilizados nesse capítulo. Pelos padrões atuais, o número de marcadores
genotípicos utilizado é bastante pequeno (cerca de 350 marcadores, 3 ordens de
grandeza a menos que no capítulo 5). Essa baixa densidade, aliadada às poucas
gerações de recombinação a partir dos fundadores, limita a resolução dos
nossos mapeamentos e a nossa habilidade de distinguir pleiotropia em sua
definição mais estrita, de um locus com efeitos na variação de vários
caracteres, de desequilíbrio de ligação entre loci não
pleiotrópicos\footnote{Que eventualmente seria quebrado por recombinação}.
Apesar disso, podemos pensar nos efeitos combinados de pleiotropia e
desequilíbrio de ligação e quantificar como esses efeitos geram a covariação,
sem tentar diferenciar os dois. Um das vantagens de um banco de dados um pouco
antigo e que já foi explorado de forma bastante completa na
literatura~\parencite{Cheverud1996-fm, Kramer1998-cc, Hager2009-mz,
Vaughn1999-wt, Leamy2002-nh, Wolf2005-nr, Mitteroecker2016-vq} é que nos
sentimos mais livres para tentar métodos novos e explorar os dados de forma
menos ortodoxa. Com isso em mente, nós utilizamos um modelo de mapeamento
genético multivariado relativamente incomum, ajustando os efeitos aditivos e
de dominância em todos os caracteres simultaneamente para cada marcador. Além
disso, exploramos também uma regressão regularizada, que permite ajustar o
modelo aditivo e de dominância para todos os marcadores
simultaneamente.\footnote{Os resultados são semelhantes entre os dois métodos,
e destacamos aqui que ajustar esse tipo de modelo sem utilizar programas
escritos especialmente para o problema em questão seria impensável há alguns
anos. Nós implementamos todos os modelos em programas genéricos para o ajuste
de modelos lineares. Esse processo em si foi um aprendizado, e nosso sucesso
mostra o quão poderosas são as ferramentas estatísticas que temos disponíveis
hoje.} A partir dos vetores de efeitos aditivos e de dominância, nós mostramos
como a divergência entre os fundadores se deve principalmente a efeitos
aditivos, e, utilizando somente os vetores aditivos estimados na população
recombinante, fomos capazes de prever de forma bastante satisfatória o
fenótipo dos fundadores. Os vetores aditivos carregam o sinal de seleção, e
vetores maiores são mais alinhados com a direção de seleção e divergência
fenotípica entre fundadores.  Em oposição aos efeitos aditivos, os efeitos de
dominância não apresentam nenhum sinal de alinhamento com a seleção. Essa
diferença provavelmente se deve em parte aos efeitos aditivos serem um
resultado do processo seletivo que levou à diferenciação entre as linhagens
fundadoras: os efeitos aditivos observados na F$_3$ são resultado da seleção,
os efeitos de dominância não. Estes últimos são um fenômeno emergente do
encontro de alelos que evoluíram separados, e o efeito de sua interação não
foi moldado por seleção.

Além da diferença no alinhamento com seleção, os efeitos aditivos e de
dominância também diferem na sua relação com o padrão modular. Como os efeitos
aditivos são mais alinhados com seleção, poderíamos esperar que a distribuição
de efeito aditivos tivesse o primeiro componente de variação entre efeitos
alinhado com a seleção. Apesar disso, a distribuição de efeitos aditivos
recupera o padrão modular entre as fases precoce e tardia do crescimento, com
as duas principais direções de variação dos efeitos aditivos alinhadas com as
duas fases do crescimento. Uma hipótese que poderíamos levantar para explicar
esse padrão modular é uma influência do desenvolvimento na expressão dos
efeitos aditivos, que de alguma forma poderia restringir o padrão de
pleiotropia. Mas, se esse fosse o caso, a mesma restrição também se aplicaria
aos efeitos de dominância. Mas a distribuição dos efeitos de dominância não
apresenta nenhum padrão modular. Aqui, com nossos dados, só podemos especular,
mas podemos novamente pensar na relação entre a história dos efeitos aditivos,
que são moldados por seleção, e o processo de desenvolvimento, que difere nas
fases precoce e tardia do crescimento. Juntos essas duas influências podem
explicar a diferença entre os efeitos aditivos e de dominância, pois o efeito
da seleção precisa de alguma forma ser processado pelo
desenvolvimento~\parencite{Klingenberg2008-ll}. Ao mesmo tempo, os efeitos de
dominância são expressos pelo desenvolvimento, mas de forma emergente e menos
relacionada com a história evolutiva da população.

A relação entre os efeitos pleiotrópicos e a modularidade é explorada de forma
direta no capítulo 4, e nós estimamos a matriz de covariância esperada dados
os efeitos pleiotrópicos detectados pelo mapeamento genético. Essa
matriz esperada é bastante similar à matriz genética estimada pela
similaridade entre irmãos completos, e nós mostramos como as matrizes devido
aos efeitos aditivos e de dominância tem padrões modulares diferentes,
reforçando o efeito na população das diferenças exploradas acima. A combinação
desses efeitos distintos forma a covariância observada na F$_3$.

O estudo da ligação entre efeitos pleiotrópicos e covariação é continuada no
capítulo 5, que funciona como uma extensão do capítulo 4, mas que procura
abordar o mesmo problema de outra perspectiva. Esse capítulo utiliza um banco
de dados mais moderno, de 33300 marcadores e uma geração mais avançada
de um intercruzamento de camundongos mais complexo. Esse intercruzamento foi
gerado utilizando fundadores vindos de um experimento de seleção artificial
para a alteração da curva de crescimento, com seleção divergente sendo
aplicada nas fases precoce e tardia. O resultado deste regime seletivo é uma
mudança generalizada na taxa de crescimento ao longo de todo o desenvolvimento
dos animais. A mistura de seis linhagens com curvas de crescimento muito
divergentes gerou uma população ultra variável em sua curva de crescimento, e
que apresenta um padrão de covariação entre as fases do crescimento bastante
diferente do padrão ancestral, antes da seleção. A estrutura familiar mais
complexa desses dados cria algumas dificuldades metodológicas no mapeamento
genético, pois para levar o efeito do parentesco entre indivíduos em conta no
mapeamento genético é necessário estimar um modelo animal para cada marcador.
Existem métodos aproximados bastante eficientes para a realização desse
mapeamento, como o FaST-LMM~\parencite{Lippert2011-jn} ou o
GEMMA~\parencite{Zhou2012-zl}, mas a maioria desses programas se limita a
estimativa de efeitos aditivos, ignorando os efeitos de dominância. Neste
capítulo, por restrições de tempo e por essa restrição do programas eficientes
de mapeamento, nós nos limitamos a explorar a contribuição aditiva dos efeitos
pleiotrópicos para a covariação. Além da população experimental, nós também
desenvolvemos um modelo de simulação computacional para a evolução de
caracteres quantitativos controlados por genes pleiotrópicos, no qual o padrão
de pleiotropia pode evoluir de forma contínua. A partir deste modelo, nós
criamos algumas expectativas qualitativas para a distribuição de efeitos
pleiotrópicos em diversos cenários de seleção estabilizadora e direcional.
Essas expectativas são baseadas na classificação dos vetores de efeitos
pleiotrópicos em categorias de acordo a direção do efeito em relação ao um
padrão modular pré-definido. As simulações sugerem que no cenário de seleção
divergente, como foi o caso dos fundadores da população intercruzada, haveria
um aumento na proporção de alelos que afetam as duas fases do crescimento em
direções opostas, um padrão pleiotrópico que nós chamamos de antagonista.
Esses efeitos antagonistas seriam os principais responsáveis pelas correlações
positivas dentro de cada fase do crescimento e pelas correlações negativas
entre as fases. De fato, no mapeamento de loci quantitativos na geração F$_6$
do intercruzamento nós encontramos uma série de efeitos antagonistas, mas,
diferente da simulações, não mais do que as outras classes de efeitos,
modulares, integrativos e intra-modulares. A matriz de correlação genética
esperada dados todos os efeitos que conseguimos detectar é relativamente
parecida com a matriz de covariância genética estimada a partir de um modelo
animal. Apesar disso, as matrizes esperadas para cada classe de efeito
pleiotrópico não é compatível com a nossa expectativa ao criar a classe. Por
exemplo, os efeitos modulares, em tese, deveriam contribuir para a correlação
dentro de módulos, mas a matriz esperada a partir dos efeitos modulares
apresenta correlações altas entre módulos. Da mesma forma, os efeitos
integradores deveriam gerar correlações positivas entre todos os caracteres,
mas a matriz esperada a partir desses efeitos apresenta correlações negativas.
A coerência do padrão global e os problemas dos padrões dentro de cada classe
sugere que nossas estimativas de cada um dos efeitos pleiotrópicos é ruidosa
demais para que a classificação seja feita de forma confiável, mas o padrão
global recupera algum sinal biológico. Nossa habilidade de caracterizar as
origens do padrão de covariação observado também foi prejudicada pela ausência
do componentes de dominância, que, especialmente nessa população, tem um
grande potencial de contribuir para a covariância\footnote{Isso porque a
estrutura dos cruzamentos garante que um alelo presente em um dos fundadores
vai estar em frequência próxima de 1/6 na geração F$_6$, e quanto mais longe
da frequência de 1/2, mais o componente dominante contribui para a
covariação.}. Apesar dessa deficiência, o estudo dos efeitos aditivos ilustra
como o padrão de covariância é formado por contribuições de diversos tipos de
efeitos pleiotrópicos que interagem de forma complexa.

\subsubsection{Próximos passos}

As analises dos capítulos 4 e 5 se limitam a caracterizar a distribuição de
efeitos pleiotrópicos presentes nas populações intercruzadas. Essa
distribuição é interpretada à luz das nossas expectativas teóricas e
relacionada à história evolutiva dos fundadores das duas populações. Os
vetores de efeitos pleiotrópicos são também relacionados ao padrão de
covariação entre os caracteres. Essa caracterização fina da distribuição de
efeitos pleiotrópicos nos informa sobre as origens da covariação genética, e
da pistas sobre a distribuição de efeitos mutacionais e sobre os processos de
desenvolvimento que dão origem a esses efeitos pleiotrópicos. O que nossa
analise não fornece é uma quantificação da variação nesses efeitos. Para isso,
seria necessária a inclusão de efeitos epistáticos nos modelos de mapeamento,
e essa é uma extensão fundamental de toda nossa análise. Animais vindos do
mesmo cruzamento de camundongos utilizado do capítulo 4 já foram usados para o
estudo do efeito de interações gênicas em diversos contextos, como uma
quantificação da contribuição de interações epistáticas na relação alométrica
entre ossos longos e peso~\parencite{Pavlicev2008-jy}, para avaliar a
contribuição de epistasia para a covariância entre caracteres cranianos que se
desenvolvem em fases diferentes do desenvolvimento ~\parencite{Wolf2005-nr}, e
para quantificar variação na intensidade das correlações entre caracteres
mandibulares~\parencite{Cheverud2004-qr}. Esses exemplos ilustram a
contribuição importante dos efeitos epistáticos para a covariação aditiva, e,
principalmente, a contribuição dos efeitos epistáticos na variação dos efeitos
pleiotrópicos. 

Para dar continuidade ao estudo da evolução da covariação,
pretendemos investigar a relação entre os efeitos epistáticos e a história
seletiva das linhagens em um conjunto mais diverso de caracteres. Na população
intercruzada do capítulo 5, que foi criada a partir de seis linhagens
fundadoras, nós também medimos uma série de fenótipos ligados à composição
corporal e ao sistema esquelético como um todo, incluindo medidas de alta
resolução do crânio. Todos esses fenótipos são divergentes entre as linhagens
fundadoras, e de formas mais complicadas que simplesmente diferenças em
tamanho corporal\footnote{Os cruzamentos entre as linhagens \textit{Large} e
\textit{Small}, utilizados no capítulo 4 e por boa parte da literatura em
efeitos genéticos e covariação, sofrem do problema da simplicidade da
divergência entre os fundadores. É difícil investigar padrão mais elaborados
de pleiotropia na morfologia se a maioria dos efeitos está alinhado com a
divergência em tamanho e a maioria dos efeitos pleiotrópicos é sinergístico.}.
Uma vantagem de ampliar a analise para outros fenótipos é que estes podem ser
cada vez mais afastados do alvo de seleção. Aqui nós analisamos a arquitetura
genética dos caracteres sob seleção, e isso tem consequências para a o padrão
de efeitos pleiotrópicos, como vimos principalmente no capitulo 4. Quando nos
afastamos do alvo de seleção, ainda observamos divergência, mas essa
divergência pode potencialmente ser menos estruturada pela seleção e fornecer
padrões diferentes de epistasia e pleiotropia. Como a variação de efeitos
pleiotrópicos depende das interações epistáticas, essas diferenças trazem
consequências para a evolução da covariância entre caracteres, como foi
discutido brevemente no capitulo 3. Estudar a variação epistática em conjuntos
caracteres cada fez mais distantes do alvo de seleção pode potencialmente
trazer uma gama maior de organizações pleiotrópicas e de padrões de
covariação. 

O estudo de caracteres cranianos na população do capítulo 5 também fornece um
sistema controlado para entender como as mudanças no padrão de integração
observados no crânios do capítulo 3 efetivamente acontecem ao nível de efeitos
pleiotrópicos. Os resultados em~\textcite{Porto2016-qc} sugere que essas
mudanças são devido ao número de alelos pleiotrópicos de efeito generalizado,
e a escala de tempo em que nós observamos as mudanças de integração sugerem
qualquer mudança nos efeitos pleiotrópicos deve ser baseada em variação
epistática, pois não haveria tempo de acumular mutações pleiotrópicas
suficientes para explicar a mudança de integração observada no capítulo 3.
Além disso, a a população intercruzada do capítulo 5 apresenta uma curva de
crescimento muito variável, e portanto se presta ao estudo da relação entre
integração no crânio e taxa de crescimento, dois caracteres relacionados por
um padrão macroevolutivo claro~\parencite{Porto2013-dc}. Esse padrão
macroevolutivo associa taxas de crescimento (ou investimento energético no
crescimento) elevado com taxas de integração elevadas em diversos táxons.
Nosso banco de dados permite verificar se esse padrão macroevolutivo se mantém
no nível microevolutivo e tentar entender os processos que levam a essa
associação.

Um resultado consistente ao longo de todos os nossos capítulo é a presença de
variação no padrão dos efeitos pleiotrópicos e, consequentemente, a habilidade
do padrão de covariação de mudar em resposta a pressões seletivas. A
intensidade das covariâncias muda de forma bastante rápida sob seleção no
experimento do capítulo 3, e o padrão de correlação foi alterado por seleção
nos fundadores do capítulo 5. Os efeitos pleiotrópicos detectados nos
capítulos 4 e 5 podem potencialmente gerar padrões de covariância diferentes
somente com alterações das frequências alélicas, mesmo quando ignoramos as
interações epistáticas. Essa maleabilidade dos padrões de covariância era
sugerida por modelos teóricos~\parencite{Barton1989-ag, Arnold2008-pc}, e
realmente nós temos uma literatura considerável documentando mudanças
significativas no padrão de covariância de diversos tipos de caracteres
medidos em populações naturais, numa escala de tempo
ecológica~\parencite{Merila1996-rw, Doroszuk2008-qe, Bjorklund2013-io,
Pfrender2000-if, Eroukhmanoff2011-ph}, e até em escala de tempo
macroevolutivo, associadas a mudanças de hábito
locomotor~\parencite{Young2005-nk, Young2010-rm}. Exitem alguns problemas
metodológicos com esse tipo de estudo em populações naturais, principalmente
devido à dificuldade em se obter amostras grandes o suficiente aliada à
dificuldade inerente em se estimar uma matriz de covariância genética de forma
suficientemente precisa~\parencite{Marroig2012-jd}. Devido a esses problemas
de estimativa, é bastante difícil diferenciar mudanças biologicamente
relevantes de mudanças significativas mas irrelevantes, principalmente porque
a maioria dos métodos de estimativa da matriz de covariância genética aditiva
não fornecem boas estimativas do intervalo de confiança para as
covariâncias\footnote{As aproximações envolvidas em calcular o erro padrão de
um parâmetro de variância num modelo animal ajustado por máxima
verossimilhança são bastante duvidosas, como por exemplo a simetria da
distribuição do parâmetro. Principalmente no caso relativamente comum de um
parâmetro ser estimado na borda do seu intervalo de suporte, em que a
distribuição do parâmetro é completamente assimétrica.} e métodos eficientes
para estimar esses intervalos são relativamente
incipientes~\parencite{Houle2015-jb, Runcie2013-nr}. De qualquer forma, todos
esses resultados apontam para um quadro de mudanças rápidas de covariação e de
variação abundante para o padrão e intensidade de covariâncias. Por outro
lado, varias estimativas de padrões de covariância estimados em populações
naturais apresentam uma notável estabilidade numa escala macroevolutiva
~\parencite{Porto2009-pi, Marroig2001-ne, McGlothlin2018-hm, Steppan1997-oa,
Garant2008-wv} mesmo com a facilidade observada em se modificar o padrão de
covariância em curto prazo. A questão da estabilidade macroevolutiva da matriz
de covariância genética é fundamental para o estudo de diversificação, pois
uma matriz G estável permite reconstruir o padrão de seleção
ancestral~\parencite{Jones2004-be} e implica uma possível restrição
macroevolutiva à resposta seletiva, principalmente em superfícies adaptativas
com muitos picos, como discutimos no capítulo 6. Essa inconsistência entre
resultados de curto e longo prazo sugere que existem forças evolutivas
mantendo o padrão de covariância estável em escalas de tempo longas, e algumas
propostas foram feitas nesse sentido, como estabilidade dos padrões gerais de
alocação de recursos~\parencite{Bjorklund2004-tp}, ou processos de seleção
estabilizadora interna~\parencite{Cheverud1984-mi}, relacionados à
compatibilidade entre caracteres dentro do mesmo organismo. Outro mecanismo
possível para a manutenção da matriz G em longas escalas de tempo é a
constância da matriz de covariância mutacional, pois mesmo que flutuações de
curto prazo alterem a matriz G, novas mutações podem reestabelecer o padrão
variacional. Existe alguma evidência para um alinhamento entre matrizes
mutacionais e genéticas~\parencite{Houle2017-ph}, mas ainda não sabemos como
esse mecanismo para a manutenção da covariações se comporta sob seleção e como
ele se manifesta no padrão de efeitos pleiotrópicos. Esperamos dar
continuidade no estudo da evolução das covariações genéticas nesse contexto
macroevolutivo, e entender os mecanismos que regem a interação entre
restrições genéticas e os processos evolutivos.


\printbibliography


\end{refsection}