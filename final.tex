\pagestyle{empty}
\cleardoublepage
\pagestyle{fancy}

\selectlanguage{brazilian}

\chapter{Considerações Finais}

\begin{refsection}

Nesta tese abordamos o problema da evolução e manutenção dos padrões de
covariância genética aditiva em caracteres complexos. Para isso, utilizamos
seleção artificial em diversos desenhos experimentais. Todas os regimes de
seleção artificial envolveram crescimento ou tamanho corporal. A complexidade
da definição do tamanho de um organismo garante que seleção neste caráter irá
ter influências indiretas em praticamente todos os caracteres morfológicos da
população, e portanto as consequências da seleção dependem criticamente da
arquitetura genética e do padrão de covariação. Neste capítulo final, vamos
discutir brevemente os resultados de cada um dos três capítulos experimentais
e tecer alguns comentários sobre as limitações e possíveis futuros
desdobramentos do nosso trabalho.

A influência da seleção para tamanho no resto do organismo é bastante clara no
capítulo 3. Neste capítulo, mostramos como seleção para alterações no peso,
tanto para aumento quanto diminuição, causam respostas correlacionadas nos
caracteres cranianos. A alteração coordenada dos caracteres do crânio leva a
uma mudança na estrutura da covariação genética e fenotípica. A mudança na
covariação genética tem dois aspectos importantes. Primeiro, a quantidade de
variação genética total diminui. Esse é o tipo de mudança esperado sob seleção
de curto prazo em caracteres de base genética aditiva. A seleção provoca
mudanças nas frequências alélicas dos loci envolvidos na variação dos
caracteres sob seleção. A medida que os alelos que conferem maior aptidão
aumentam de frequência e eventualmente se fixam, a variância genética dos
caracteres sob seleção diminui. Na seleção de caracteres multivariados, a
predição clássica para a mudança da covariância sob seleção direcional é que o
efeito seja o mesmo, com a redução da variação longo da direção sob seleção. O
livro texto clássico \textcite{Falconer1996-ot} discute a mudança esperada na
correlação entre dois caracteres selecionados na mesma direção. A expectativa
nesse caso é que, ao final do processo seletivo, os alelos que influenciam os
dois caracteres na mesma direção estejam fixados, e que portanto apenas alelos
com efeitos na direção ortogonal à seleção estariam segregando em frequências
intermediárias. Como só haveria variação na direção ortogonal aos dois
caracteres mudando juntos, a correlação entre eles ao final do processo
deveria ser negativa. Isso é o exato oposto do que nós observamos no crânio
dos camundongos selecionados. A perda de variação total que nós observamos é
completamente anisotrópica, e algumas direções perdem mais variação do que
outras. Mas, ao contrario do esperado pelo modelo aditivo, a direção que perde
menos variação é justamente a direção de seleção. Proporcionalmente à variação
total, a seleção direcional aumenta a proporção de variação na direção de
seleção. Nós interpretamos esse resultado à luz da teoria recente de genética
quantitativa que inclui efeitos não aditivos na determinação do padrão de
covariação~\parencite{Pavlicev2011-wz, Cheverud2001-ho}. A inclusão de efeitos
epistáticos no nosso modelo da evolução da covariância genética da conta das
mudanças na covariância observadas nas linhagens selecionadas. Apesar disso,
nossa estimativa da matriz de covariância genética é bastante deficiente
devido à estrutura familiar das linhagens sob seleção, uma consequência do
desenho experimental que não foi feito com esse tipo de medida em
mente\footnote{O experimento de seleção artificial foi originalmente concebido
em um departamento de ciências veterinárias, e o objetivo do estudo era
avaliar a viabilidade das linhagens após seleção de longo prazo, com
aplicações para melhoramento genético bovino. Mérito do Gabriel em perceber a
possibilidade de aproveitar a colonia para o estudo da evolução da
covariação.}, e portanto boa parte dos resultados depende de algum nível de
similaridade entre as matrizes genéticas e fenotípicas, que parece razoável
dada nossa experiência com esses caracteres do
crânio~\parencite{Garcia2014-oj, Cheverud1988-he}. Além disso, não temos
nenhum dado de marcadores genéticos para esses animais, o que limita nossa
habilidade de confirmar a origem genética das mudanças na covariação. Mesmo
com essas limitações, esses resultados ilustram bem o tipo de mudança bastante
não intuitiva que seleção direcional pode ter na covariação genética e
fenotípica num intervalo de tempo relativamente curto.

O capítulo 4 utiliza um cruzamento clássico~\parencite{Cheverud1996-fm} de
camundongos selecionados para alterações no tamanho corporal para investigar o
padrão pleiotrópico e a covariação genética entre fases do crescimento. Este
capítulo tem dois temas principais: (1) a caracterização dos efeitos
pleiotrópicos em relação à seleção e a divergência fenotípica entre os
fundadores; (2) a inferência dos padrões de covariação a partir dos efeitos
aditivos e de dominância. Cabe aqui um comentário sobre os dados utilizados
nesse capítulo. Pelos padrões atuais, o número de marcadores genotípicos
utilizado é bastante pequeno (cerca de 350 marcadores, 3 ordens de grandeza a
menos que no capítulo 5). Essa baixa densidade aliadada às poucas gerações de
recombinação a partir dos fundadores limita a resolução dos nossos mapeamentos
e a nossa habilidade de distinguir pleiotropia em sua definição mais estrita,
de um locus com efeitos na variação de vários caracteres, de desequilíbrio de
ligação entre loci não pleiotrópicos\footnote{Que eventualmente seria quebrado
por recombinação}. Apesar disso, podemos pensar nos efeitos combinados de
pleiotropia e desequilíbrio de ligação e quantificar como esses efeitos geram
a covariação, sem tentar diferenciar os dois. Um das vantagens de um banco de
dados um pouco antigo e que já foi explorado de forma bastante completa na
literatura~\parencite{Cheverud1996-fm, Kramer1998-cc, Hager2009-mz,
Vaughn1999-wt, Leamy2002-nh, Wolf2006-xt, Wolf2005-nr, Mitteroecker2016-vq} é
que nos sentimos mais livres para tentar métodos novos e explorar os dados de
forma menos ortodoxa. Com isso em mente, nós utilizamos um modelo de
mapeamento genético multivariado relativamente incomum, ajustando os efeitos
aditivos e de dominância em todos os caracteres simultaneamente para cada
marcador. Além disso, exploramos também uma regressão regularizada, que
permite ajustar o modelo aditivo e de dominância para todos os marcadores
simultaneamente.\footnote{Os resultados são semelhantes entre os dois métodos,
e destacamos aqui que ajustar esse tipo de modelo sem utilizar programas
escritos especialmente para o problema em questão seria impensável há alguns
anos. Nós implementamos todos os modelos em programas genéricos para o ajuste
de modelos lineares. Esse processo em si foi um aprendizado, e nosso sucesso
mostra o quão poderosas são as ferramentas estatísticas que temos disponíveis
hoje.} A partir dos vetores de efeitos aditivos e de dominância, nós mostramos
como a divergência entre os fundadores se deve principalmente a efeitos
aditivos, e, utilizando somente os vetores aditivos estimados na população
recombinante, fomos capazes de prever de forma bastante satisfatória o
fenótipo dos fundadores. Os vetores aditivos carregam o sinal de seleção, e
vetores maiores são mais alinhados com a direção de seleção e divergência.  Em
oposição aos efeitos aditivos, os efeitos de dominância não apresentam nenhum
sinal de alinhamento com a seleção. Essa diferença provavelmente se deve em
parte aos efeitos aditivos serem um resultado do processo seletivo que levou à
diferenciação entre as linhagens fundadoras: os efeitos aditivos observados na
F$_3$ são resultado da seleção, os efeitos de dominância não. Estes últimos
são um fenômeno emergente do encontro de alelos que evoluíram separados, e o
efeito de sua interação não foi moldado por seleção.

Além da diferença no alinhamento com seleção, os efeitos aditivos e de
dominância também diferem na sua relação com o padrão modular. Como os efeitos
aditivos são mais alinhados com seleção, poderíamos esperar que a distribuição
de efeito aditivos tivesse o primeiro componente de variação entre efeitos
alinhado com a seleção. Apesar disso, a distribuição de efeitos aditivos
recupera o padrão modular entre as fases precoce e tardia do crescimento, com
as duas principais direções de variação dos efeitos aditivos alinhadas com as
duas fases do crescimento. Um hipotese que poderíamos levantar para explicar
esse padrão modular é uma influência do desenvolvimento na expressão dos
efeitos aditivos, que de alguma forma poderia restringir o padrão de
pleiotropia. Mas, se esse fosse o caso, a mesma restrição também se aplicaria
aos efeitos de dominância. Mas a distribuição dos efeitos de dominância não
apresenta nenhum padrão modular. Aqui, com nossos dados, só podemos especular,
mas podemos novamente pensar na relação entre a história dos efeitos aditivos,
que são moldados por seleção, e o processo de desenvolvimento, que difere nas
fases precoce e tardia do crescimento. Juntos essas duas influências podem
explicar a diferença entre os efeitos aditivos e de dominância, pois o efeito
da seleção precisa de alguma forma ser processado pelo
desenvolvimento~\parencite{Klingenberg2008-ll}. Ao mesmo tempo, os efeitos de
dominância são expressos pelo desenvolvimento, mas de forma emergente e menos
relacionada com a história evolutiva da população.

A relação entre os efeitos pleiotrópicos e a modularidade é explorada de forma
direta no capítulo 4, e nós estimamos diretamente a matriz de covariância
esperada dados os efeitos pleiotrópicos que nós estimamos usando o mapeamento
genético. Essa matriz esperada é bastante similar com a matriz genética
estimada pela similaridade entre irmãos completos, e nós mostramos como as
matrizes devido aos efeitos aditivos e de dominância tem padrões modulares
diferentes, reforçando o efeito na população das diferenças exploradas acima.
A combinação desses efeitos distintos forma a covariância observada na F$_3$.

O estudo da ligação entre efeitos pleiotrópicos e covariação é continuada no
capítulo 5, que funciona como uma extensão do capítulo 4. Esse capítulo
utiliza um banco de dados muito mais moderno, de 33300 marcadores e uma
geração mais avançada de um intercruzamento mais complexo. A estrutura
familiar mais complexa desses dados cria algumas dificuldades metodológicas no
mapeamento genético, pois para levar o efeito do parentesco entre indivíduos
em conta no mapeamento genético é necessário estimar um modelo animal para
cada marcador. Existem métodos aproximados bastante eficientes para a
realização desse mapeamento, como o FaST-LMM~\parencite{Lippert2011-jn} ou o
GEMMA~\parencite{Zhou2012-zl}, mas a maioria desses programas se limita a
estimativa de efeitos aditivos, ignorando os efeitos de dominância. Neste
capítulo, por restrições de tempo e por essa restrição do programas eficientes
de mapeamento, nós nos limitamos a explorar a contribuição aditiva dos efeitos
pleiotrópicos para a covariação.




- Lições:

	- seleção altera covariação
	- o mecanismo dessa alteração é complexo
		- padrões modulares podem ou não ser mantidos
		- Padrão de seleção importa na manutenção do padrão pleiotrópico
	- Precisamos de estudos em mais sistemas para ter uma visão mais completa
	- Incluir coisas não aditivas é fundamental

- Estabilidade de covariação em tempo evolutivo é realmente impressionante. 




\printbibliography


\end{refsection}