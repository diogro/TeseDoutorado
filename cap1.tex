% Faz com que o ínicio do capítulo sempre seja uma página ímpar
\cleardoublepage

% Inclui o cabeçalho definido no meta.tex
\pagestyle{fancy}

% Números das páginas em arábicos
\pagenumbering{arabic}

\chapter{Introdução}\label{intro}


\section{Motivação}\label{intro:historico}

\begin{refsection}

Esta tese surgiu da vontade de entender como os processos internos ao
organismo que produzem a variação em uma população interagem com os processos
evolutivos para dar origem à diversidade que observamos na natureza.
Resumidamente, nós procuramos estudar como a covariação genética se estabelece
e como ela evolui, e situamos essas questões dentro do contexto da
macroevolução de caracteres quantitativos. Para isso, nos valemos da teoria de
genética quantitativa na sua encarnação mais moderna, aliada a teoria de
evolução e das técnicas de mapeamento de loci de caracteres quantitativos.
Essa combinação permite estudar e prever as consequências das restrições
genéticas na evolução dos caracteres, e esmiuçar a arquitura genética por traz
dessas restrições. Nesta introdução, vamos revisar rapidamente a teoria de
genética quantitativa evolutiva e discutir a estrutura geral da tese. Neste
capítulo vamos utilizar as citações de forma esparsa, de forma a deixar o texto mais
fluido. A maior parte do contexto teórico pode ser encontrado em livros
básicos de evolução e genética, como~\textcite{Falconer1996-ot,
Lynch1998-ql, Barton2007-hq}. Alguns trechos mais avançados podem ser
encontrados em~\textcite{Rice2004-jf, Buerger2000-ez}. Argumentos que surgiram
em artigos específicos e são mais associados a essa referencia serão
apontados no texto.

\section{Genética quantitativa} 



\printbibliography
\end{refsection}