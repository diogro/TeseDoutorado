% Faz com que o ínicio do capítulo sempre seja uma página ímpar
\cleardoublepage

% Inclui o cabeçalho definido no meta.tex
\pagestyle{fancy}

% Números das páginas em arábicos
\pagenumbering{arabic}

\chapter{Introdução}\label{intro}


\section{Motivação}\label{intro:historico}

\begin{refsection}

Esta tese surgiu da vontade de entender como os processos internos ao
organismo que produzem a variação em uma população interagem com os processos
evolutivos para dar origem à diversidade que observamos na natureza.
Resumidamente, nós procuramos estudar como a covariação genética se estabelece
e como ela evolui, e situamos essas questões dentro do contexto da
macroevolução de caracteres quantitativos. Para isso, nos valemos da teoria de
genética quantitativa na sua encarnação mais moderna, aliada a teoria de
evolução e das técnicas de mapeamento de loci de caracteres quantitativos.



\printbibliography
\end{refsection}